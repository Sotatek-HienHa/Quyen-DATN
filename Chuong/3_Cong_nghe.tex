\documentclass[../DoAn.tex]{subfiles}
\begin{document}
Chương này trình bày các công nghệ, nền tảng và giao thức được sử dụng trong quá trình xây dựng hệ thống giám sát vị trí cho thiết bị nhúng NaviTracker. Việc lựa chọn công nghệ được thực hiện dựa trên yêu cầu chức năng và phi chức năng đã phân tích ở Chương 2, đảm bảo khả năng thu thập dữ liệu định vị GNSS, truyền thông ổn định tới nền tảng giám sát từ xa, và hiển thị thông tin một cách trực quan, linh hoạt.

Cụ thể, chương sẽ giới thiệu về nền tảng phần cứng của thiết bị NaviTracker, các giao thức truyền thông được tích hợp để đảm bảo trao đổi dữ liệu hiệu quả, và nền tảng ThingsBoard được sử dụng làm hệ thống hiển thị thông tin. Mỗi công nghệ được phân tích về nguyên lý hoạt động, lý do lựa chọn và vai trò của nó trong hệ thống tổng thể. Đồng thời, chương cũng sẽ so sánh một số lựa chọn thay thế có thể, qua đó làm rõ sự phù hợp của các công nghệ được sử dụng trong đồ án.
\section{Thiết bị NaviTracker}
\label{section:3.1}
Thiết bị NaviTracker là một nền tảng phần cứng nhúng được thiết kế chuyên biệt cho mục đích giám sát vị trí trong thời gian thực. Thiết bị tích hợp nhiều thành phần chính bao gồm: mô-đun vi điều khiển trung tâm SC20, mô-đun GNSS để thu thập dữ liệu định vị toàn cầu, mô-đun truyền thông không dây để truyền dữ liệu tới hệ thống giám sát ThingsBoard, cùng bộ nhớ trong để lưu tạm thời dữ liệu khi mất kết nối mạng. Nhờ sự tích hợp này, thiết bị có khả năng hoạt động độc lập, ghi nhận và truyền dữ liệu định vị một cách ổn định và liên tục.

Vi điều khiển SC20 đóng vai trò như một bộ xử lý trung tâm của hệ thống. Nó chịu trách nhiệm thu nhận dữ liệu GNSS từ mô-đun định vị thông qua giao tiếp UART, sau đó xử lý các bản tin theo định dạng chuẩn như NMEA hoặc UBX. Sau khi xử lý, SC20 trích xuất các thông tin cần thiết như tọa độ hiện tại, tốc độ di chuyển, hướng, và thời gian. Các dữ liệu này được SC20 định dạng lại và gửi lên máy chủ ThingsBoard bằng giao thức MQTT để hiển thị trên giao diện người dùng. Ngoài ra, SC20 cũng có khả năng thực hiện các tính toán cục bộ như xác định quãng đường di chuyển, phát hiện trạng thái đứng yên, kiểm tra vùng địa lý (geofence), và đánh giá tính hợp lệ của dữ liệu GNSS.

Một đặc điểm nổi bật trong thiết kế của NaviTracker là sự bổ sung của vi điều khiển phụ ATmega8 nhằm tối ưu hóa mức tiêu thụ năng lượng của toàn hệ thống. ATmega8 hoạt động như một bộ quản lý năng lượng thông minh, có nhiệm vụ điều khiển việc bật và tắt SC20 dựa trên trạng thái phát hiện chuyển động. Khi cảm biến phát hiện có chuyển động ở đầu vào (kết nối tại chân PD3), ATmega8 sẽ kích hoạt SC20 bằng cách điều khiển chân PWRKEY trong một chu kỳ nhất định để bật nguồn SC20. Đồng thời, một đèn LED trạng thái cũng được bật để báo hiệu thiết bị đang hoạt động. Sau một khoảng thời gian nhất định nếu không còn chuyển động nào được phát hiện, ATmega8 sẽ đưa SC20 trở về trạng thái tắt nhằm tiết kiệm điện năng, đồng thời hệ thống được đưa vào chế độ ngủ sâu bằng chế độ sleep.

Kiến trúc này mang lại nhiều lợi ích thực tiễn. Thứ nhất, việc sử dụng ATmega8 giúp hạn chế đáng kể mức tiêu thụ năng lượng trong những thời điểm thiết bị không cần hoạt động liên tục, qua đó kéo dài thời gian sử dụng pin. Thứ hai, SC20 chỉ được kích hoạt khi có nhu cầu thu thập và truyền dữ liệu, nhờ đó duy trì được độ ổn định và tránh hao phí tài nguyên hệ thống. Cơ chế bật/tắt thông minh này phù hợp với các ứng dụng ngoài thực địa, nơi nguồn năng lượng thường bị giới hạn và thiết bị cần hoạt động bền bỉ trong thời gian dài.

So với các lựa chọn thay thế như sử dụng các bo mạch phát triển đa năng (Arduino, Raspberry Pi), giải pháp sử dụng NaviTracker thể hiện rõ ưu điểm về độ ổn định, tính gọn nhẹ, khả năng tích hợp truyền thông di động sẵn có, và đặc biệt là hiệu suất tiêu thụ năng lượng vượt trội. Điều này cho phép thiết bị dễ dàng triển khai trong các hệ thống IoT phân tán, nơi yêu cầu khắt khe về độ tin cậy, khả năng tự chủ và thời gian vận hành lâu dài.

Việc tích hợp hài hòa giữa SC20 và ATmega8 giúp đảm bảo tính liên tục trong thu thập và truyền tải dữ liệu định vị ngay cả khi thiết bị không được giám sát thường xuyên. Cấu trúc phần cứng này là nền tảng vững chắc cho các chức năng giám sát hành trình, xác định vùng địa lý, và phân tích di chuyển được trình bày trong các chương sau.
\section{Ngôn ngữ và môi trường phát triển}
\label{section:3.2}
Ứng dụng được phát triển cho thiết bị NaviTracker sử dụng hệ điều hành Android, với ngôn ngữ lập trình chính là Java. Đây là một lựa chọn phổ biến trong phát triển phần mềm di động và thiết bị nhúng chạy Android, nhờ vào hệ sinh thái phong phú, tính ổn định cao và khả năng truy cập trực tiếp các API phần cứng thông qua Android SDK.

Việc lựa chọn Android làm nền tảng phát triển xuất phát từ yêu cầu về khả năng xử lý linh hoạt, dễ dàng tương tác với mô-đun GNSS qua giao tiếp UART, và khả năng tích hợp các thư viện hỗ trợ giao thức MQTT để truyền dữ liệu lên hệ thống giám sát ThingsBoard. Môi trường phát triển sử dụng là Android Studio – một công cụ tích hợp hiện đại do Google phát triển, hỗ trợ đầy đủ cho việc lập trình, biên dịch, mô phỏng và gỡ lỗi ứng dụng Android.

So với các ngôn ngữ khác như C/C++ vốn thường dùng trong các hệ thống vi điều khiển truyền thống, Java trên Android mang lại lợi thế rõ rệt về tính mô-đun, khả năng mở rộng và dễ bảo trì mã nguồn. Đồng thời, hệ điều hành Android cho phép triển khai đồng thời nhiều tác vụ, như thu thập dữ liệu GNSS, lưu trữ cục bộ và truyền tải qua mạng, một cách hiệu quả nhờ vào cơ chế quản lý tiến trình và luồng xử lý riêng biệt.

Việc sử dụng Java và Android giúp rút ngắn thời gian phát triển, đồng thời đáp ứng linh hoạt các yêu cầu được phân tích trong Chương 2, đặc biệt là khả năng truyền dữ liệu định vị lên server và tương tác theo thời gian thực với các thành phần của hệ thống giám sát từ xa.

Song song với phần mềm Android, một chương trình nhúng chạy trên vi điều khiển ATmega8 cũng được phát triển nhằm quản lý nguồn điện cho thiết bị SC20. Chương trình này được viết bằng ngôn ngữ lập trình C, sử dụng bộ thư viện chuẩn AVR-GCC. Đây là trình biên dịch mã nguồn mở phổ biến cho dòng vi điều khiển AVR, hỗ trợ đầy đủ các chức năng truy cập thanh ghi, điều khiển ngắt ngoài, chế độ ngủ và thời gian trễ chính xác.

Việc kết hợp Java (trên Android) và C (trên ATmega8) tạo thành một hệ thống nhúng đa tầng, trong đó mỗi thành phần đảm nhận nhiệm vụ riêng biệt nhưng phối hợp chặt chẽ. Trong khi phần mềm Android xử lý các tác vụ cao cấp như phân tích GNSS, truyền dữ liệu MQTT và phản hồi theo thời gian thực, thì chương trình C trên ATmega8 đảm nhận việc tiết kiệm năng lượng bằng cách điều khiển chu kỳ hoạt động của thiết bị một cách thông minh.

Cách tiếp cận này mang lại sự cân bằng giữa tính linh hoạt của hệ điều hành Android và hiệu quả năng lượng của vi điều khiển nhúng, đảm bảo hệ thống hoạt động bền bỉ trong môi trường thực tế với yêu cầu cao về độ tin cậy và thời gian vận hành liên tục.

\section{Giao thức truyền thông MQTT}
\label{section:3.3}
MQTT (Message Queuing Telemetry Transport) là một giao thức truyền thông dạng publish/subscribe, được thiết kế đặc biệt cho các hệ thống nhúng và thiết bị IoT có tài nguyên giới hạn. Với đặc tính nhẹ, độ trễ thấp và cơ chế giữ kết nối ổn định, MQTT phù hợp để truyền dữ liệu trong thời gian thực, đặc biệt trong các môi trường mạng không ổn định hoặc có băng thông hạn chế.

Giao thức này hoạt động trên nền TCP/IP và cho phép các thiết bị gửi (publish) bản tin đến các chủ đề (topic) cụ thể, trong khi các bên nhận (subscriber) chỉ cần đăng ký nhận dữ liệu từ những chủ đề quan tâm. Cấu trúc này giúp hệ thống linh hoạt, dễ mở rộng và giảm đáng kể độ phức tạp khi tích hợp nhiều thiết bị trong cùng một mạng lưới truyền thông.

Một trong những lý do chính để lựa chọn MQTT là khả năng duy trì kết nối liên tục thông qua cơ chế keep-alive, giúp đảm bảo truyền dữ liệu ổn định trong thời gian dài mà không cần thiết lập lại kết nối cho mỗi phiên làm việc. Trong môi trường mạng không ổn định, MQTT có thể tự động phát hiện mất kết nối và tiến hành kết nối lại, đồng thời hỗ trợ gửi lại bản tin nếu quá trình truyền bị gián đoạn. Những đặc điểm này giúp nâng cao độ tin cậy trong quá trình truyền dữ liệu thời gian thực giữa thiết bị và hệ thống trung tâm.

So với các giao thức khác thường được sử dụng trong hệ thống nhúng, MQTT thể hiện rõ ưu thế. Cụ thể: 
HTTP tuy phổ biến nhưng yêu cầu thiết lập kết nối lại cho mỗi lần truyền và hoạt động theo mô hình client-server nên gây tiêu tốn tài nguyên, không phù hợp với thiết bị chạy nền hoặc yêu cầu tiêu thụ năng lượng thấp. 

CoAP là một lựa chọn nhẹ hơn khi sử dụng nền UDP, nhưng lại thiếu các cơ chế kiểm soát chất lượng dịch vụ và dễ gặp vấn đề mất gói trong mạng không ổn định. 

WebSocket cung cấp giao tiếp hai chiều theo thời gian thực, tuy nhiên không định nghĩa rõ cấu trúc bản tin, thiếu cơ chế xử lý lỗi, kiểm soát luồng, và không hỗ trợ các tính năng đặc thù như mức QoS hay bản tin cuối cùng (last will) như MQTT.

Với khả năng vận hành hiệu quả trên cả nền tảng hệ điều hành đầy đủ như Android lẫn các hệ thống nhúng tối giản, MQTT phù hợp để triển khai trong các ứng dụng yêu cầu truyền dữ liệu định kỳ, giám sát liên tục và xử lý sự kiện trong thời gian thực. Thư viện mã nguồn mở như Eclipse Paho hoặc Mosquitto giúp việc tích hợp MQTT trở nên dễ dàng trong các môi trường phát triển hiện đại, cho phép ứng dụng nhanh chóng thiết lập kết nối, gửi và nhận bản tin với độ trễ thấp và độ tin cậy cao.

Tổng kết lại, MQTT là lựa chọn tối ưu cho hệ thống truyền thông của thiết bị nhúng giám sát vị trí, nhờ sự cân bằng giữa hiệu suất truyền tải, độ ổn định, và mức tiêu thụ tài nguyên thấp. Việc lựa chọn giao thức này là nền tảng quan trọng đảm bảo hệ thống hoạt động hiệu quả trong điều kiện thực tế, nơi thiết bị cần vận hành liên tục, ổn định và tiêu thụ năng lượng tiết kiệm.

\section{Nền tảng hiển thị: ThingsBoard}
\label{section:3.4}
ThingsBoard là một nền tảng mã nguồn mở được phát triển nhằm cung cấp giải pháp toàn diện cho việc quản lý thiết bị IoT, thu thập và xử lý dữ liệu từ xa, cũng như hiển thị dữ liệu một cách trực quan và có hệ thống. Đây là một trong những nền tảng phổ biến nhất hiện nay trong lĩnh vực IoT, đặc biệt được ưa chuộng trong các ứng dụng yêu cầu giám sát thời gian thực, phản ứng tức thì với sự kiện, và quản trị thiết bị với quy mô lớn. Kiến trúc của ThingsBoard được thiết kế theo mô hình phân tán, hỗ trợ khả năng mở rộng theo chiều ngang, cho phép xử lý đồng thời hàng nghìn thiết bị và luồng dữ liệu mà vẫn đảm bảo hiệu năng ổn định.

Một trong những ưu điểm nổi bật của ThingsBoard là khả năng tương thích với nhiều giao thức truyền thông chuẩn trong lĩnh vực IoT, bao gồm MQTT, CoAP và HTTP. Điều này cho phép nền tảng dễ dàng kết nối với đa dạng thiết bị từ nhiều nhà sản xuất khác nhau, mà không yêu cầu phải thay đổi phần mềm hay phần cứng sẵn có của thiết bị. Giao thức MQTT – vốn được tối ưu cho các môi trường mạng không ổn định hoặc băng thông hạn chế - đặc biệt phù hợp để triển khai với ThingsBoard trong các ứng dụng giám sát từ xa hoặc trong môi trường công nghiệp.

Về mặt hiển thị dữ liệu, ThingsBoard cung cấp hệ thống dashboard động, cho phép người dùng thiết kế giao diện giám sát tùy chỉnh mà không cần lập trình. Các widget được hỗ trợ bao gồm bản đồ tương tác, biểu đồ dạng tuyến, thanh trạng thái, bảng dữ liệu, đồng hồ đo, biểu tượng cảnh báo, v.v. Nhờ khả năng kéo - thả và cấu hình trực quan, người vận hành hệ thống có thể dễ dàng tạo ra các giao diện phù hợp với mục đích giám sát, trình bày dữ liệu theo thời gian thực hoặc truy xuất dữ liệu lịch sử một cách có tổ chức.

Không chỉ dừng lại ở việc thu thập và hiển thị dữ liệu, ThingsBoard còn hỗ trợ xử lý logic sự kiện thông qua hệ thống rule engine - một thành phần cốt lõi cho phép xây dựng các chuỗi xử lý phức tạp theo nguyên tắc điều kiện - hành động. Ví dụ, hệ thống có thể được cấu hình để tự động phát hiện sự kiện bất thường như mất tín hiệu định vị, vượt ngưỡng tốc độ, thay đổi trạng thái hoạt động hoặc rời khỏi khu vực địa lý được phép, và ngay lập tức sinh ra cảnh báo hoặc gửi thông báo đến người quản trị. Việc xử lý logic này được thực hiện trực tiếp trên server, giúp thiết bị đầu cuối giảm bớt gánh nặng xử lý, đồng thời đảm bảo tính thống nhất và độ tin cậy của hệ thống phản hồi.

Bên cạnh đó, ThingsBoard cũng cung cấp khả năng tương tác hai chiều giữa thiết bị và nền tảng quản lý, cho phép hệ thống không chỉ tiếp nhận dữ liệu mà còn gửi cấu hình điều khiển ngược lại thiết bị từ phía máy chủ. Điều này đặc biệt quan trọng trong các ứng dụng cần thay đổi thông số vận hành động, ví dụ như cập nhật chu kỳ gửi dữ liệu, điều chỉnh vùng giám sát hoặc bật/tắt một tính năng cụ thể mà không cần can thiệp thủ công trực tiếp vào thiết bị. Cơ chế này giúp giảm thiểu thời gian vận hành, nâng cao tính linh hoạt và cho phép điều khiển tập trung toàn bộ hệ thống từ một giao diện duy nhất.

So sánh với các nền tảng tương tự, ThingsBoard thể hiện ưu thế rõ ràng về mức độ tích hợp và khả năng mở rộng. Blynk là một giải pháp đơn giản hơn, dễ tiếp cận cho người mới bắt đầu nhưng lại thiếu tính năng xử lý logic, không hỗ trợ hiển thị dữ liệu lịch sử và khó mở rộng khi triển khai ở quy mô lớn. Node-RED cung cấp công cụ mạnh để xây dựng luồng xử lý dữ liệu bằng hình ảnh nhưng lại không có hệ thống dashboard phù hợp sẵn cho các tác vụ giám sát trực quan. Grafana là nền tảng mạnh về trực quan hóa dữ liệu chuỗi thời gian nhưng không được thiết kế dành riêng cho thiết bị IoT và thiếu khả năng tương tác hai chiều. Trong khi đó, ThingsBoard tích hợp tất cả các thành phần cần thiết – từ thu thập, xử lý, lưu trữ, đến hiển thị và điều khiển - trong một hệ thống duy nhất, mang lại trải nghiệm nhất quán và dễ triển khai hơn trong thực tiễn.

Với mã nguồn mở, tài liệu phong phú và cộng đồng phát triển rộng lớn, ThingsBoard không chỉ thích hợp cho các ứng dụng thương mại mà còn là lựa chọn lý tưởng trong môi trường học thuật, nghiên cứu và phát triển nguyên mẫu. Nền tảng này có thể triển khai trên nhiều loại hạ tầng khác nhau, từ máy chủ cục bộ đến môi trường điện toán đám mây, hỗ trợ đầy đủ các công nghệ hiện đại như Docker, Kubernetes và cơ sở dữ liệu thời gian thực như PostgreSQL hoặc Cassandra.

Việc lựa chọn ThingsBoard nhằm đáp ứng trực tiếp các yêu cầu đã phân tích trong Chương 2 như: hiển thị tọa độ hiện tại, ghi lại lịch sử di chuyển, tính toán quãng đường di chuyển, thiết lập vùng geofence và gửi cảnh báo. Nhờ khả năng cấu hình linh hoạt và khả năng mở rộng, nền tảng này phù hợp để triển khai cả trong môi trường thử nghiệm lẫn ứng dụng thực tế ngoài hiện trường.
\section{Các thư viện và phần mềm hỗ trợ}
\label{section:3.5}
Trong quá trình phát triển ứng dụng giám sát vị trí cho thiết bị NaviTracker, một số thư viện và phần mềm hỗ trợ đã được sử dụng nhằm đơn giản hóa việc lập trình, nâng cao độ ổn định của hệ thống, và rút ngắn thời gian triển khai. Các thành phần này chủ yếu phục vụ cho các chức năng thu thập dữ liệu GNSS, xử lý bản tin, truyền thông qua giao thức MQTT, và hiển thị thông tin tại nền tảng ThingsBoard.

Trên phía thiết bị, thư viện Eclipse Paho MQTT Client được sử dụng để hiện thực kết nối MQTT trong môi trường Android. Thư viện này hỗ trợ đầy đủ các chức năng publish/subscribe, quản lý kết nối, và xử lý sự kiện mạng, giúp quá trình truyền dữ liệu từ thiết bị lên server diễn ra ổn định và tin cậy. Ngoài ra, các thư viện truy cập UART như android-serialport-api hỗ trợ thiết bị đọc dữ liệu GNSS trực tiếp từ cổng nối tiếp, trích xuất thông tin định vị theo chuẩn NMEA.

Về phía ThingsBoard, các công cụ tích hợp sẵn như Rule Engine, Attribute Mapping, và Widget Library hỗ trợ xử lý dữ liệu đầu vào và trực quan hóa thông tin trên dashboard. Các widget được sử dụng để hiển thị tọa độ GPS, lịch sử di chuyển, geofence, cũng như trạng thái hoạt động của thiết bị theo thời gian thực.

So với việc tự xây dựng các thành phần này từ đầu, việc sử dụng thư viện và phần mềm có sẵn giúp đảm bảo tính ổn định và tương thích với hệ thống, đồng thời phù hợp với thực tiễn triển khai các ứng dụng IoT hiện nay. Tất cả các thư viện sử dụng đều là mã nguồn mở và được cộng đồng hỗ trợ rộng rãi, đảm bảo dễ dàng bảo trì và mở rộng trong tương lai.
\section{Tổng kết chương}
\label{section:3.6}
Trong chương này, tôi đã trình bày các công nghệ, nền tảng và công cụ chính được sử dụng trong quá trình phát triển hệ thống giám sát vị trí NaviTracker. Cụ thể, phần cứng thiết bị NaviTracker được lựa chọn nhờ khả năng tích hợp GNSS và truyền thông di động, đáp ứng yêu cầu giám sát vị trí liên tục. Về phần mềm, ứng dụng được phát triển trên nền Android với ngôn ngữ Java, cho phép triển khai linh hoạt các chức năng như thu thập dữ liệu, truyền thông qua MQTT, và xử lý cục bộ. Giao thức MQTT được sử dụng để đảm bảo việc truyền dữ liệu nhanh, nhẹ và ổn định trong môi trường IoT. Nền tảng ThingsBoard đóng vai trò trung tâm hiển thị, lưu trữ và tương tác với thiết bị thông qua các dashboard trực quan. Cuối cùng, các thư viện và phần mềm hỗ trợ giúp đơn giản hóa việc lập trình, nâng cao độ tin cậy và khả năng mở rộng của hệ thống.

Tất cả các công nghệ trên đều được lựa chọn dựa trên yêu cầu thực tiễn của hệ thống đã phân tích ở Chương 2, đồng thời được cân nhắc kỹ lưỡng so với các lựa chọn thay thế. Việc tích hợp các thành phần này một cách hợp lý đã tạo nền tảng kỹ thuật vững chắc cho việc hiện thực hóa hệ thống trong các chương tiếp theo.
\end{document}