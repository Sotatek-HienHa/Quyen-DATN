\documentclass[../DoAn.tex]{subfiles}
\begin{document}

\section{Đặt vấn đề}
\label{section:1.1}
Trong những năm gần đây, công nghệ định vị toàn cầu đã phát triển mạnh mẽ và ngày càng được ứng dụng rộng rãi trong nhiều lĩnh vực khác nhau. Các thiết bị định vị được tích hợp trong ô tô, thiết bị vận tải, thiết bị giám sát cá nhân, và hệ thống quản lý tài sản ngày càng phổ biến. Tuy nhiên, việc giám sát vị trí một cách chính xác và liên tục vẫn đang gặp nhiều thách thức như độ ổn định của tín hiệu vệ tinh, khả năng quản lý dữ liệu thu được, và sự hạn chế của các thiết bị giám sát truyền thống trong việc tương tác với nền tảng quản lý từ xa.

Thực tế hiện nay, nhu cầu giám sát và quản lý phương tiện vận tải, tài sản hoặc con người trong thời gian thực ngày càng tăng cao, đặc biệt là trong các ngành logistics, an toàn giao thông, quản lý đội xe, và cả trong giám sát người già, trẻ nhỏ. Việc xác định được chính xác vị trí và tình trạng hoạt động của đối tượng theo dõi sẽ giúp tối ưu hóa các hoạt động quản lý, nâng cao hiệu quả vận hành, giảm thiểu rủi ro và tiết kiệm chi phí vận hành cho doanh nghiệp cũng như cá nhân.

Nếu giải quyết được vấn đề giám sát và quản lý vị trí một cách liên tục và chính xác, các doanh nghiệp và cá nhân không chỉ được hưởng lợi từ việc cải thiện đáng kể hiệu quả quản lý và an toàn, mà công nghệ này còn có thể mở rộng ứng dụng sang các lĩnh vực khác như quản lý đô thị thông minh, giám sát môi trường, và hỗ trợ khẩn cấp trong các tình huống thiên tai. Từ đó, việc nghiên cứu và phát triển một hệ thống giám sát vị trí với độ chính xác và tin cậy cao trở thành vấn đề cấp thiết và có ý nghĩa thực tiễn quan trọng hiện nay.

\section{Mục tiêu và phạm vi đề tài}
\label{section:1.2}
Hiện nay, với sự phát triển mạnh mẽ của công nghệ định vị toàn cầu (GNSS) và các thiết bị IoT, việc giám sát hành trình, quản lý phương tiện, và đảm bảo an toàn giao thông ngày càng được chú trọng. Nhiều sản phẩm đã được phát triển nhằm đáp ứng nhu cầu theo dõi và quản lý vị trí theo thời gian thực, trong đó có các thiết bị định vị GPS tích hợp trên phương tiện, ứng dụng giám sát từ xa qua internet, và các hệ thống quản lý dữ liệu tập trung. Các thiết bị này thường được thiết kế để ghi lại tọa độ, tốc độ, và lộ trình di chuyển, đồng thời gửi dữ liệu lên máy chủ để quản lý và theo dõi.

Tuy nhiên, các hệ thống hiện tại còn tồn tại một số hạn chế như độ chính xác thấp trong điều kiện tín hiệu yếu, dễ bị tấn công giả mạo tín hiệu GNSS, hoặc thiếu khả năng tùy chỉnh phù hợp với các yêu cầu cụ thể của người dùng. Bên cạnh đó, một số giải pháp giám sát vị trí hiện tại chưa tích hợp được tính năng phát hiện và cảnh báo khi có tín hiệu giả mạo, dẫn đến nguy cơ mất an toàn và thiếu tin cậy trong các ứng dụng quan trọng.

Dựa trên các phân tích trên, đồ án này tập trung vào việc phát triển một thiết bị giám sát hành trình tích hợp, có khả năng giám sát vị trí và phát hiện tín hiệu GNSS giả mạo. Thiết bị sẽ được tối ưu về hiệu năng xử lý và mức tiêu thụ năng lượng để đảm bảo phù hợp với các điều kiện hoạt động khắc nghiệt của IoT. Thiết bị không chỉ thu thập tọa độ GNSS, mà còn thực hiện phân tích tín hiệu để phát hiện những bất thường hoặc nguy cơ tấn công giả mạo, từ đó cung cấp thông tin giám sát chính xác và đáng tin cậy cho người sử dụng.

Mục tiêu chính của đề tài là xây dựng một hệ thống giám sát GNSS tối ưu, có khả năng hoạt động ổn định trong điều kiện khắc nghiệt, đồng thời đảm bảo tính an toàn và chính xác của dữ liệu định vị. Hệ thống sẽ được tích hợp lên nền tảng ThingsBoard để hiển thị vị trí hiện tại, lộ trình di chuyển, và các cảnh báo về tín hiệu giả mạo trong thời gian thực. Kết quả của đề tài không chỉ nâng cao tính tin cậy trong giám sát vị trí, mà còn góp phần vào việc bảo vệ an toàn cho các ứng dụng yêu cầu tính chính xác cao trong quản lý phương tiện và tài sản.

\section{Định hướng giải pháp}
\label{section:1.3}
Dựa trên nhiệm vụ được xác định trong phần trước, đồ án này lựa chọn hướng tiếp cận sử dụng các công nghệ GNSS, xử lý tín hiệu số và nền tảng IoT ThingsBoard để giải quyết bài toán giám sát hành trình và phát hiện tín hiệu GNSS giả mạo. Công nghệ GNSS được sử dụng để xác định vị trí thời gian thực, trong khi xử lý tín hiệu số sẽ giúp phân tích và phát hiện các tín hiệu bất thường. Nền tảng IoT ThingsBoard sẽ đảm nhiệm vai trò hiển thị thông tin vị trí, quản lý dữ liệu, và phát cảnh báo trực quan. Việc lựa chọn các công nghệ này dựa trên khả năng đáp ứng yêu cầu về độ chính xác, tính ổn định và khả năng mở rộng của hệ thống.

Giải pháp của đồ án tập trung vào việc thiết kế một thiết bị giám sát hành trình GNSS tích hợp, với khả năng nhận diện tín hiệu giả mạo. Thiết bị sẽ thu thập dữ liệu GNSS từ các cảm biến, xử lý tín hiệu để phát hiện các bất thường, và gửi dữ liệu vị trí cùng cảnh báo về hệ thống giám sát ThingsBoard. Người dùng có thể theo dõi lộ trình, kiểm tra tính xác thực của tín hiệu định vị và nhận cảnh báo ngay lập tức khi có dấu hiệu đáng ngờ.

Định hướng cụ thể của giải pháp bao gồm: xây dựng phần mềm cho thiết bị IoT giám sát hành trình GNSS có khả năng chống giả mạo tín hiệu và tích hợp hệ thống vào nền tảng ThingsBoard để hiển thị và quản lý dữ liệu theo thời gian thực. Kết quả đạt được là một hệ thống giám sát hành trình an toàn, đáng tin cậy, đáp ứng nhu cầu quản lý vị trí và bảo vệ dữ liệu định vị trong các ứng dụng thực tế.
\section{Bố cục đồ án}
\label{section:1.4}
Phần còn lại của báo cáo đồ án tốt nghiệp này được tổ chức như sau:

Chương 2 trình bày về khảo sát và phân tích yêu cầu của hệ thống giám sát GNSS tích hợp. Nội dung chương tập trung vào việc thu thập thông tin từ các tài liệu nghiên cứu, các giải pháp hiện có, cũng như khảo sát nhu cầu thực tiễn của người dùng đối với việc giám sát hành trình và bảo vệ tín hiệu GNSS. Chương này cũng đưa ra phân tích các yêu cầu chức năng và phi chức năng của hệ thống, từ đó xây dựng các tiêu chí để đánh giá hiệu quả của thiết bị giám sát GNSS tích hợp.

Chương 3 giới thiệu về các công nghệ và công cụ được sử dụng trong quá trình phát triển hệ thống. Các công nghệ chính bao gồm GNSS cho định vị, xử lý tín hiệu số để phát hiện giả mạo, và ThingsBoard để quản lý và hiển thị dữ liệu. Ngoài ra, chương này cũng mô tả ngắn gọn về các thư viện và nền tảng hỗ trợ triển khai giải pháp như MQTT, xử lý tín hiệu DSP và các giao thức truyền nhận dữ liệu từ thiết bị IoT lên server.

Chương 4 tập trung vào thiết kế phần mềm cho thiết bị giám sát GNSS tích hợp. Chương này trình bày kiến trúc hệ thống, mô hình phân lớp của phần mềm và các mô-đun chức năng chính. Các phương pháp tối ưu hiệu năng và tiết kiệm năng lượng cũng được thảo luận trong chương này nhằm đảm bảo thiết bị có thể hoạt động ổn định trong điều kiện khắc nghiệt.

Chương 5 mô tả việc triển khai server để thu thập, xử lý và hiển thị dữ liệu từ thiết bị giám sát. Hệ thống server được xây dựng dựa trên nền tảng ThingsBoard với khả năng hiển thị vị trí theo thời gian thực, lưu trữ lịch sử di chuyển, và quản lý các cảnh báo. Các vấn đề liên quan đến bảo mật và độ tin cậy của hệ thống cũng được phân tích trong chương này.

Chương 6 đưa ra kết luận về kết quả đạt được trong quá trình nghiên cứu và phát triển hệ thống giám sát GNSS tích hợp. Chương này tổng hợp lại các đóng góp chính của đề tài, đồng thời phân tích những điểm mạnh và hạn chế của hệ thống. Bên cạnh đó, chương này cũng đề xuất các hướng phát triển trong tương lai, bao gồm việc nâng cao độ chính xác của thuật toán phát hiện giả mạo và tích hợp thêm các chức năng thông minh vào hệ thống.
\end{document}